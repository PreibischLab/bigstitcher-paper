%\section{SUPPLEMENTARY METHODS}
\titleformat{\section}{\centering\normalfont\fontsize{13.5pt}{1em}\bfseries}{SUPPLEMENTARY NOTE \thesection: }{0em}{}
\section*{SUPPLEMENTARY NOTES}
\titleformat{\section}{\centering\normalfont\fontsize{11.5pt}{1em}\bfseries}{\thesection. }{0em}{}

\hspace{20mm}

\section{Data Import}
\label{sec:sup-note1}

\subsection*{\texttt{SpimData} data format}

We internally represent our image data and metadata using an extended version of the \texttt{SpimData} data format of BigDataViewer\cite{pietzsch2015bigdataviewer}. Each image stack is defined by a (\texttt{ViewSetup}, \texttt{TimePoint})-combination. We extend the format by giving each \texttt{ViewSetup} the following \emph{attributes}: \texttt{Channel} to represent color channels, \texttt{Illumination} to represent illumination directions, \texttt{Angle} to represent multi-view acquisition angles and finally \texttt{Tile}, representing (local) x,y points in a multipoint acquisition.

In addition to those attributes, we store detected interest points, bounding boxes (named sub-volumes in which we can \emph{fuse} or deconvolve images), point spread functions for deconvolution and pairwise registrations (that have yet to be used in global optimization) for each (\texttt{ViewSetup}, \texttt{TimePoint}) \emph{view}. For each image stack, we also store its \emph{registration} (i.e. the transformation from pixel to world coordinates) as a list of affine transform matrices. The registration steps described below will typically prepend another transformation matrix to this list. Finally, the \texttt{SpimData} is associated with an \texttt{ImgLoader} object that can make image pixel data available as an ImgLib2 \cite{imglib2} \texttt{RandomAccessibleInterval} given a  (\texttt{ViewSetup}, \texttt{TimePoint}) \emph{view id}.

The \texttt{SpimData} data structure can be saved as an XML \emph{project file}, allowing users to manually edit it with any text editor. We automatically save previous versions of the project file to provide the user with the ability to un-do registration steps.

\subsection*{Import of data}

BigStitcher imposes little constraints on the format and naming of raw data files. Using the Bioformats \cite{bioformats} library, we support a large variety of image file formats, from "conventional" TIFF stacks to vendor-specific formats. The assignment of attributes to the image stacks in the raw data can be done automatically or with minimal interaction from the users. We offer the possibility to immediately compute multi-resolution pyramids from the images and saving them as chunks to HDF5 files. Furthermore, making use of cached ImgLib2 data structures, we support virtual loading of image planes from the raw files with caching of already loaded planes.

\begin{comment}
\subsection{Illumination selection}

When imaging large samples with multiple illumination directions,  a lot of unnecessary images are acquired since typically, only illumination from one direction provides optimal images. We therefore implemented a simple \emph{illumination selection} functionality in BigStitcher. It starts by \emph{combining} all (selected) images by their \texttt{Illumination} attribute, i.e. it groups images that share all other attributes besides \texttt{Illumination}. In each of the resulting groups we select a best image. We do this by loading the pixel data for all images in the group at the lowest resolution level (in the case of non-multiresolution images, this corresponds to the original image) and calculating a \emph{quality metric}. We currently offer mean intensity and mean gradient magnitude as quality metrics. The image with the highest score is kept, while all other images are marked as \emph{missing} in the \texttt{SpimData}, which will lead to them being ignored in subsequent processing steps. 
\end{comment}

\section{Flat-field correction}
\label{sec:flatfield}

Flat-field correction is the process of correcting for image artifacts due to uneven illumination or detection efficiency or fixed-pattern noise. Aside from being visually unpleasing, especially in tiled acquisitions, these artifacts can also effect image registration and downstream quantitative image analyses. We therefore offer simple on-the-fly correction for a \emph{dark image} (which might be nonzero due to e.g. camera offset) and a \emph{bright image} (representing uneven ilumination or detection efficiency across the field-of-view). We calculate corrected pixel intensities $C$ from a raw image $R$ and bright and dark images $B$ and $D$ as:

\begin{equation}
\label{eq:flatfield-eq1}
C_{x} = \frac{(R_{x} - D_{x'}) * \overline{(B-D)} }{(B_{x'}- D_{x'})}
\end{equation}

The correction images can either have the same dimensionality as the raw images, in which case $x' = x$, or have lower dimensionality (e.g. when using 2D correction images on a 3D image stack), in which case $x' = (x_1 \hdots x_n)$ with $n$ being the dimensionality of the correction images. If a dark image is not provided by the user, we assume it to have constant intensity of 0 (corresponding to no background offset). Likewise, if no bright image is provided, we assume it to have constant intensity of 1 (uniform illumination and detection efficiency). 

We implemented the flat-field correction as a wrapper around an \texttt{ImgLoader}, calculating corrected pixel intensity values on-the-fly (with optional caching) every time an image is loaded. That way, the corrected images are available for all other processing steps such as intensity-based registration, interest point detection or image fusion, but it is still possible to activate or de-activate the correction or change bright or dark images after the initial flat-field correction. A separate (bright, dark)-correction image pair can be set for every image in the dataset by modifying the XML project file, while in the GUI we offer user-friendly assignment of correction images to every (channel, illumination direction)-pair.

\section{Pairwise shift calculation}

In BigStitcher, we currently offer three ways of calculating shifts between a pair of images: the Fourier-based \emph{phase correlation} algorithm, the Gradient-descent-based \emph{Lucas-Kanade} algorithm, both intensity-based methods, as well as interest point-based alignment.

\subsection*{Phase correlation}

By default, we calculate pairwise translational shifts of two images $I_1$ and $I_2$ using phase correlation \cite{preibisch2009globally, kuglin1975phase} as explained in the \textbf{Online Methods} section.

%an ImgLib2 re-implementation of the Fourier-based \emph{phase correlation} algorithm \cite{preibisch2009globally}. In noiseless images, the method produces a phase correlation matrix $Q$ containing a single $\delta$-impulse at the location corresponding to the shift between the two images. Real images might contain multiple peaks, so we localize the $n$ highest peaks in $Q$. By detecting peaks with subpixel accuracy using a quadratic fit \cite{lowe2004distinctive}. Aside from allowing subpixel-accurate registration, we can also use this to counteract the effects of downsampling, allowing us to perform registration of similar quality to full-resolution with significant performance gains.
%
%Due to the periodic nature of the Fourier shift theorem, each peak in the PCM can actually correspond to $2^d$ possible shifts in $d$ dimensions. We therefore test each of these candidate shifts by calculating the cross-correlation between the images with $I2$ shifted according to the candidate shift (optionally with interpolation in the case of sub-pixel shifts). In the end, we keep the shift vector $t$ corresponding the highest cross correlation as the final result (applying downsampling correction, if necessary).

\subsection*{Lucas-Kanade}

In addition to the default phase correlation-based pairwise shift calculation, we offer registration via an ImgLib2 implementation of the \emph{inverse compositional} formulation of the gradient descent-based Lucas-Kande optical flow algorithm \cite{baker2004lucas}. While the algorithm is applicable to a variety of transformation models, we currently stick to estimating a translation vector $t$. If the pairwise registration converges, we calculate the cross correlation of the overlapping portions of the images as a quality metric for the pairwise registration. 

\subsection*{Intensity-based Registration of grouped images}

In many use cases, one might want to align not single images but groups of images, e.g. all channels of a tile, in the pairwise registration step. For this, we implemented a flexible framework for the registration of grouped images.

Each attribute of the images can be set to be an \emph{axis of application}, an \emph{axis of comparison} or an \emph{axis of grouping}. The registration will proceed by first splitting the images by the application attributes, i.e. grouping all images that have the same value for these attributes. In each each group, the images are then split by the comparison attributes and finally, the remaining image groups (that differ only in the grouping attributes) are combined into one image stack by either averaging all images for each grouping attribute or picking the image with a specific instance of the attribute.   

In a typical application, the stitching of tiled datasets, we would, for example, start by \emph{applying} the registration to all (\texttt{Angle}, \texttt{TimePoint})-combinations individually, \emph{comparing} by \texttt{Tiles} and finally \emph{grouping} by \texttt{Illumination} and \texttt{Channel} for each tile, e.g. by averaging illumination directions and picking a specific channel.

\subsection*{Intensity-based registration of images with pre-registrations}

The two images $I_1$ and $I_2$ can have arbitrary pre-registrations, i.e. pixel coordinates $x_{px}$ are mapped to world coordinates $x_w$ via the affine transforms $x_{w,I_1} = A _{I_1}x_{px,I_1} + b_{I_1}$ and $x_{w,I_2} = A _{I_2}x_{px,I_2} + b_{I_2}$. Depending on the values of $ A _{I_1}$ and $ A _{I_2}$, we consider two cases: If they are equal, i.e. the pre-registrations differ only by a translation, we perform the shift calculation on the raw pixel data of the overlapping volume to get a shift vector $t$ for $I_2$ in pixel coordinates. The transformation in world coordinates is then given by $R \bigl(\begin{smallmatrix}  & I & & t \\  0 & \cdots & 0 & 1 \end{smallmatrix}\bigr) R^{-1} $ with $R = \bigl(\begin{smallmatrix}  & A_{I_2} & & b_{I_2} \\  0 & \cdots & 0 & 1 \end{smallmatrix}\bigr)$. If the pre-registrations differ in more than just translation, we create virtually transformed images of the smallest rectangular bounding box enclosing the overlapping volume and use them as input to the registration. As the virtual input images are already in world coordinates in this case, the resulting transformation matrix for $I_2$ is simply $\bigl(\begin{smallmatrix}  & I & & t \\  0 & \cdots & 0 & 1 \end{smallmatrix}\bigr)$

\subsection*{Interest-point based}
\label{sssec:ip-registration}

For interest-point based pairwise registration, we detect local extrema in either Difference-of-Gaussian or Difference-of-mean filtered images, optionally followed by subpixel refinement of the detections via a quadratic fit \cite{lowe2004distinctive}. If we are registering a pair of image \emph{groups}, the interest points of each image in the group are pooled, with optional replacement of point clusters within a user-defined radius by their center.

For each image, we apply the current (affine) registrations to the pixel-coordinate interest points and then determine \emph{candidate point matches} via descriptor matching \cite{beads, descriptor}. We then perform model-based outlier removal via the RANSAC algorithm \cite{ransac}, yielding a set of \emph{inlier point pairs}, $C_{inliers}$, and an optimal translation $t$ for $I_2$, minimizing $\sum_{(ip_1, ip_2) \in C_{inliers} }{|| ip_1 - ip_2 - t||^2}$

\section{Global optimization}
\label{ssec:globalopt}

\subsection*{Estimation of globally optimal transformations}

The pairwise registration step results in \emph{links} between image (groups) $V$ (note that since we do not use the actual image \emph{content} in the global optimization, we will refer to the images by their integer \emph{id} in this section: $V \subset \mathbb{N}$). The links can be either in the form of pairwise transformations $T^{p}$  (such that coordinates $x$ from two images $V_i$ and $V_j$ can be transformed according to $T^p_{ij} (x_{j}) = x_i$) or \emph{point correspondences} $PM$ from which such transformations can be estimated.
The pairwise registrations thus form a \emph{link graph} $(V, C)$ with edges $C = \{(i,j) \in V \times V | T^p_{ij} \in T^p \}$ between image pairs for which we could determine pairwise transformations. Simply traversing a spanning tree of the link graph and propagating the pairwise transformations can lead to the compounding of pairwise registration errors, even if the traversal is done along a \emph{minimal} spanning tree determined according to some quality metric $q_{ij}$, e.g. cross-correlation, of the pairwise registrations.

We thus make use of an algorithm for globally optimal registration by iterative minimization of square displacement of point correspondences\cite{saalfeld2010rigid, saalfeld2012elastic} for reaching a reasonable consensus in this case. This point match-based framework allows for flexible grouping and fixing of images, is applicable to, among others, time series-, chromatic channel- or view-registration and can easily be adapted to incorporate the pairwise transformations from e.g. phase correlation. The algorithm is agnostic of the transformation model (e.g. translation, affine transform,...), with the only requirement being that the model parameters can be estimated by a least-squares fit from point correspondences.

We determine the globally optimal registrations $R$ given the image (groups) $V$, pairwise links $C$, pairwise $n$-dimensional point matches $PM$ with $PM_{ij} \subset \mathbb{R}^n \times  \mathbb{R}^n$ and a set of fixed views $F \subseteq V$ by minimizing:

\begin{equation}
\label{eq:globalopt-eq1}
\argmin_{R \setminus \{R_i | V_i \in F\}} \sum_{(i,j) \in C} \bigg( \sum_{(x_{k}, y_{k}) \in PM_{ij}} || R_{i}( x_{k}) - R_{j}( y_{k}) ||^2 \bigg)
\end{equation}

Note that for all fixed views, the registration will be constrained to be the identity transformation $I$: $\forall V_i \in F: R_{i} = I$.

\subsection*{Global optimization given pairwise transfomations}

The intensity-based pairwise shift calculations do not directly give us the point correspondences we need for the global optimization step, instead the results are pairwise transformations $T^{p}$ in the form of affine transform matrices. We can, however, easily construct point correspondences by taking a set of points and transforming them with the \emph{inverse} transform (the only requirement being that the $n$-dimensional points do not all lie in a subspace of lower dimensionality of $\mathbb{R}^n$).  

Using the 3-dimensional pairwise transformations $T^{p}$ ($T^p_{ij} (x_{j}) = x_i$) between two image (groups) $V_i$ and $V_j$ given their existing registrations $R^{meta}$, we use the 8-point approximate bounding box of their overlapping region $BB_{ij}$ to construct the point correspondences: $PM_{ij} = \{ \big(bb_k, (T^p_{ij})^{-1}(bb_k) \big) | bb_k \in BB_{ij}\}$. We can then determine the globally optimal registrations $R$ by performing the minimization described above (\ref{eq:globalopt-eq1}).  

\subsection*{Global optimization with iterative link dropping}

Once the global optimization terminates due to convergence or exceeding of the maximum number of iterations, we can calculate the \emph{error} of the individual images as the average displacement of all interest points in an image to their point matches:

\begin{equation}
\label{eq:gloablopt-eq2}
e_i = \frac{\sum_{\{j: (i,j) \in C\}}  \sum_{(x_{k}, y_{k}) \in PM_{ij}} || R_{i}( x_{k}) - R_{j}( y_{k}) ||  }{ \sum_{\{j: (i,j) \in C\}} |PM_{ij}| }
\end{equation}

If the link graph $(V,C^n)$ contains links with contradicting point correspondences, stopping after one round of global optimization might leave us with unsatisfying results. In the \emph{iterative} version of the global optimization, we therefore check that both the average error of all images and the ratio of maximal and average error fall below a user-defined threshold. If these conditions are not yet met, we will proceed to iteratively remove disagreeing links from the link graph and repeat the global optimization. To do this, we first determine the link with the highest error by maximizing:

\begin{equation}
\label{eq:globalopt-eq3}
c_{worst} = \argmax_{(i,j)} \max_{(x_k, y_k) \in PM_{ij}}\bigg( (1-q_{ij})^2 \sqrt{d_{ijk}} \log_{10}\Big(\max\big(deg(i), \deg(j)\big)\Big)\bigg) 
\end{equation}

with $d_{ijk}$ denoting the distance of the $k$'th point match of the link $(i,j)$, $d_{ijk} = || R_{i}(x_k) - R_{j}(y_k) ||$, $\deg(i)$ denoting the degree (number of neighbors) of an image $V_i$ in the link graph and $q_{ij}$ being a \emph{quality metric} $\in (0,1)$ of the link, e.g. 0-truncated cross correlation. We then remove the worst link from the links ($C^{n+1} \leftarrow C^{n} \setminus c_{worst}$) and repeat the optimization step \ref{eq:globalopt-eq1} with the new link graph $(V, C^{n+1})$. The whole process is repeated until the errors fall below a user-defined threshold (in the worst case, links will be dropped until we end up with \emph{spanning trees} of the connected components in the link graph). 

\subsection*{Two-round global optimization using metadata}

If some cases, the link graph might contain multiple connected components, e.g. in datasets from screening applications, where the actual sample only occupies isolated "islands" and most images contain only background. In this case, we can only reliably determine pairwise transformations within the connected components and align images within the components in the global optimization step. We might, however, have reasonable registrations $R^{meta}$ from metadata and wish to keep as closely as possible to those if we do not have \emph{strong} links.

For this, we developed a \emph{two-round} version of the global optimization. In the first round, we determine registrations $R^{strong}$ as described above, using the graph of \emph{strong} links, i.e. links that are backed by pairwise transformations. In the second round, we determine the connected components in the $(V, C^{strong} )$ graph and a mapping $CC: \mathbb{N} \to \mathbb{N}$ from image (group) indices to  connected component indices as well as \emph{weak links} $C^{weak} = \{(i,j) \in V \times V | CC(i) \neq CC(j) \}$ between images in different components. We then determine transformations $R^{cc}$ for each connected component not containing a fixed image by minimizing:

\begin{equation}
\label{eq:globalopt-eq4}
\argmin_{R^{cc} \setminus \{ r^{cc}_{i} \in R^{cc} | CC_i \cap F \neq \emptyset \}} \sum_{(i,j) \in C^{weak}} \sum_{bb_{k} \in BB_{ij}} || R^{cc}_{CC(i)}\big( R^{strong}_{i}( bb_{k})\big) -  R^{cc}_{CC(j)}\big( R^{strong}_{j}( bb_{k}) \big) ||^2 
\end{equation}

Note that we use the corners $bb_k$ of the bounding box $BB_{ij}$ of the overlapping volume of two images $V_i$ and $V_j$ as the point correspondences. The overlap is determined according to the metadata transformations $R^{meta}$ and we essentially try to "un-do" the registrations of the first round as well as possible (while keeping the registrations \emph{within} the connected components). The final transformations $R$ are the concatenation of the registrations within the connected components with the relative transformations of the connected components: $R_{i} \leftarrow R^{cc}_{CC(i)} R^{strong}_{i}$.

\section{MultiView Registration}

For MultiView registration, e.g. registration of angles or time series stabilization, we first detect interest points in the individual images as described above (\ref{sssec:ip-registration}). Images may be grouped (and are by default if we are, e.g. registering tiled acquisitions from multiple angles for which we already aligned the tiles via an intensity-based method) according to their attributes, by pooling their interest points and optionally merging clusters.
%For registering time-series data, we offer four strategies. First, we can treat time points individually, registering only images within a time point. We can also perform interest point matching between different time points, either comparing all-to-all, all to a user-defined \emph{reference} time point or all time points within a defined range to each other.

Pairwise point correspondences can either be established by descriptor matching followed by RANSAC outlier removal, a modified version of the iterative closest point (ICP) algorithm or by simply matching the center of mass of the point clouds of both images (note that in this case the registration will be constrained to be a translation). Using the link graph $(V,C)$ and pairwise point correspondences $P_{ij}$ established thus, we calculate the final registration by performing global optimization as described above (\ref{ssec:globalopt}), optionally with iterative link removal and a second round to preserve metadata.

\section{Image Fusion}
\label{sec:fusion}

We \emph{fuse} multiple images by performing a weighted average of the raw images $I^{raw}$ transformed by their registrations $R$. Each raw image $I^{raw}_i$ has a set weight images $W_i$. For example, we allow the user to weigh the images with a cosine-shaped fade-out, de-emphasizing the artifact-prone border regions of the individual images, as well as by the approximate local entropy, to emphasize images with sharper structures in overlapping regions. Since the raw images will be evaluated at non-integer coordinates, we offer the choice between nearest-neighbor and linear interpolation. Downsampling can easily be achieved by prepending a scaling transformation to each of the registrations $R$. The intensity of the fused volume at a coordinate $x$ is given by:

\begin{equation}
\label{eq:fusion-eq1}
I^{fused}(x) ={ \frac{  \sum_{ I^{raw}_i \in I^{raw}} \Big(I^{raw}_i\big(R_i^{-1}(x)\big) * \prod_{w_j \in W_i}{w_j\big(R_i^{-1}(x)\big)\Big)}}{ \sum_{I^{raw}_i \in I^{raw}}\Big( \prod_{w_j \in W_i}{w_j\big(R_i^{-1}(x)\big)\Big)}}}
\end{equation}

In practice, we evaluate $I^{fused}$ only at integer coordinates of a user-defined \emph{bounding box}. We implemented the image fusion to perform all calculations virtually on-the-fly, with caching of previously computed planes using imglib2-cache. This allows the quick inspection of fusion results as well as creation and planewise saving of images that might exceed the RAM available to the user.

\section{Brightness and Contrast adjustment}
\label{sec:brightness-adjust}

Even after correcting for fixed-pattern noise (\ref{sec:flatfield}), differences in brightness and contrast between images, e.g. due to bleaching, might persist and be visible in the fused images. To correct for this, we estimate optimal linear transforms of pixel intensities in adjacent images \cite{blasse2017premosa} to achieve uniform brightness and contrast in the whole dataset. We minimize the intensity difference of all pixels in the overlapping volume $O_{AB}$ of two images $I_A, I_B$ (with corresponding coordinates $(x_A, x_B)$ according to the current registrations): 

\begin{equation}
\label{eq:brighness-adjust-1}
\argmin_{\alpha, \beta} \sum_{I_A  \in I}  \bigg( \sum_{I_B \in I \setminus I_A} \Big( \sum_{(x_A, x_B) \in O_{AB}} \big( I_B(x_B) - \left[ \alpha^{I_A} I_A(x_A) + \beta^{I_A} \right] \big)^2 \Big)\bigg)
\end{equation}

Since this is equal to one-dimensional point correspondence registration, we can make use of the same iterative optimization algorithm used for image registration (\ref{ssec:globalopt}). To reduce influence of noise and computational costs, we use (precomputed) downsampled versions of the images for the optimization. A problem with unconstrained optimization is the possibility of convergence to the trivial solution of setting all pixel intensities to zero. We therefore formulate the linear transform $I(x)*\alpha + \beta$ as a weighted average between a linear transform, an additive transform and the identity transform:

\begin{equation}
\label{eq:brighness-adjust-2}
\alpha I(x) + \beta = \lambda_1 * (\alpha' I(x) + \beta_1) + \lambda_2 * (I(x) + \beta_2) + \lambda_3 * I(x) 
\end{equation}

with user-definable regularization parameters $\lambda_1, \lambda_2, \lambda_3: \lambda_1+ \lambda_2+\lambda_3 = 1$. By using nonzero $\lambda_2, \lambda_3$, we can constrain the optimization to not converge to the trivial solution.

The size of overlaps between image tiles can differ significantly. Therefore, intensity transformations supported by many overlapping pixels will implicitly have a higher weight, which can lead to the fact that visible intensity differences between tiles with little overlap persist. To compensate this effect we allow to balance overlaps by setting a maximal number of corresponding pixels. To ensure equal distribution of these corresponding pixels, we randomly remove pixels from the set of all pixels until the desired number is achieved.

\begin{comment}
\section{MultiView Deconvolution}

In addition to image fusion, we offer deconvolution of the fused volume using a MultiView formulation of the iterative Richardson-Lucy deconvolution algorithm with Tikhonov regularization and various optimizations \cite{mvdecon, richardson, lucy}. The points spread functions necessary for deconvolution can be extracted from interest points detected in the images (e.g. when subdiffraction fluorescent beads were incorporated with the sample) or supplied as TIFF-stacks by the user. We offer GPU acceleration of the deconvolution on CUDA-capable Nvidia GPUs.

\section{Headless operation}

In addition to the graphical user interface (GUI), we offer ImageJ \texttt{Plugin}s for most of the individual steps, such as data import, illumination selection, pairwise shift calculation, link filtering, global optimization and image fusion/deconvolution. The results will not be displayed interactively but saved to the XML project file or output files immediately. The individual steps can be recorded as ImageJ \emph{macros} \cite{imagej, macro-operation} and easily combined into a script for headless batch processing. 
\end{comment}
