\section{SUPPLEMENTARY METHODS}
\label{sec:sup-methods}

\subsection{\texttt{SpimData} data format}

We internally represent our image data and metadata using an extended version of the \texttt{SpimData} data format of BigDataViewer\cite{pietzsch2015bigdataviewer}. Each image stack is defined by a (\texttt{ViewSetup}, \texttt{TimePoint})-combination. We extend the format by giving each \texttt{ViewSetup} the following \emph{attributes}: \texttt{Channel} to represent color channels, \texttt{Illumination} to represent illumination directions, \texttt{Angle} to represent multi-view acquisition angles and finally \texttt{Tile}, representing (local) x,y points in a multipoint acquisition.

In addition to those attributes, we store detected interest points, bounding boxes (named sub-volumes in which we can \emph{fuse} or deconvolve images), point spread functions for deconvolution and pairwise registrations (that have yet to be used in global optimization) for each(\texttt{ViewSetup}, \texttt{TimePoint}) \emph{view}. For each image stack, we also store its \emph{registration} (i.e. the transformation from pixel to world coordinates) as a list of affine transform matrices. The registration steps described below will typically prepend another transformation matrix to this list. Finally, the \texttt{SpimData} is associated with an \texttt{ImgLoader} object that can make image pixel data available as an ImgLib2 \texttt{RandomAccessibleInterval} given a  (\texttt{ViewSetup}, \texttt{TimePoint}) \emph{view id}.

The \texttt{SpimData} data structure can be saved as an XML \emph{project file}, allowing users to manually edit it with any text editor. We automatically save previous versions of the project file to provide the user with the ability to un-do registration steps.

\subsection{Import of data}

\subsection{Illumination selection}

When imaging large samples with multiple illumination directions,  a lot of unnecessary images are acquired since typically, only illumination from one direction provides optimal images. We therefore implemented a simple \emph{illumination selection} functionality in BigStitcher. It starts by \emph{combining} all (selected) images by their \texttt{Illumination} attribute, i.e. it groups images that share all other attributes besides \texttt{Illumination}. In each of the resulting groups we select a best image. We do this by loading the pixel data for all images in the group at the lowest resolution level (in the case of non-multiresolution images, this corresponds to the original image) and calculating a \emph{quality metric}. We currently offer mean intensity and mean gradient magnitude as quality metrics. The image with the highest score is kept, while all other images are marked as \emph{missing} in the \texttt{SpimData}, which will lead to them being ignored in subsequent processing steps. 

\subsection{Flat-field correction}

Flat-field correction is the process of correcting for image artifacts due to uneven illumination or detection efficiency or fixed-pattern noise. Aside from being visually unpleasing, especially in tiled acquisitions, these artifacts can also effect image registration and downstream quantitative image analyses. We therefore offer simple on-the-fly correction for a \emph{dark image} (which might be nonzero due to e.g. camera offset) and a \emph{bright image} (representing uneven ilumination or detection efficiency across the field-of-view). We calculate corrected pixel intensities $C$ from a raw image $R$ and bright and dark images $B$ and $D$ as:

\begin{equation}
\label{eq:flatfield-eq1}
C_{x} = \frac{(R_{x} - D_{x'}) * \overline{(B-D)} }{(B_{x'}- D_{x'})}
\end{equation}

The correction images can either have the same dimensionality as the raw images, in which case $x' = x$, or have lower dimensionality (e.g. when using 2D correction images on a 3D image stack), in which case $x' = (x_1 \hdots x_n)$ with $n$ being the dimensionality of the correction images. If a dark image is not provided by the user, we assume it to have constant intensity of 0 (corresponding to no background offset). Likewise, if no bright image is provided, we assume it to have constant intensity of 1 (uniform illumination and detection efficiency). 

We implemented the flat-field correction as a wrapper around an \texttt{ImgLoader}, calculating corrected pixel intensity values on-the-fly (with optional caching) every time an image is loaded. That way, the corrected images are available for all other processing steps such as intensity-based registration, interest point detection or image fusion, but it is still possible to actviate or de-activate the correction or change bright or dark images after the initial flat-field correction. A separate (bright, dark)-correction image pair can be set for every image in the dataset by modifying the XML project file, while in the GUI we offer user-friendly assignment of correction images to every (channel, illumination direction)-pair.

\subsection{Pairwise shift calculation}

In BigStitcher, we currently offer three ways of calculating shifts between a pair of images: the Fourier-based \emph{phase correlation} algorithm, the Gradient-descent-based \emph{Lucas-Kanade} algorithm, both intensity-based methods, as well as interest point-based alignment.

\subsubsection{Phase correlation}

Due to the periodic nature of the Fourier shift theorem, each peak in the PCM can actually correspond to $2^d$ possible shifts in $d$ dimensions. We therefore test each of these candidate shifts by calculating the cross-correlation between the images with $I2$ shifted according to the candidate shift (optionally with interpolation in the case of sub-pixel shifts). In the end, we keep the shift vector $t$ corresponding the highest cross correlation as the final result (applying downsampling correction, if necessary).

\subsubsection{Lucas-Kanade}

In addition to the default phase correlation-based pairwise shift calculation, we offer registration via an ImgLib2 implementation of the \emph{inverse compositional} formulation of the gradient descent-based Lucas-Kande optical flow algorithm \cite{baker2004lucas}. While the algorithm is applicable to a variety of transformation models, we currently stick to estimating a translation vector $t$. If the pairwise registration converges, we calculate the cross correlation of the overlapping portions of the images as a quality metric for the pairwise registration. 

\subsubsection{Intensity-based Registration of grouped images}

\subsubsection{Intensity-based registration of images with arbitrary pre-registrations}

The two images $I_1$ and $I_2$ can have arbitrary pre-registrations, i.e. pixel coordinates $x_{px}$ are mapped to world coordinates $x_w$ via the affine transforms $x_{w,I_1} = A _{I_1}x_{px,I_1} + b_{I_1}$ and $x_{w,I_2} = A _{I_2}x_{px,I_2} + b_{I_2}$. Depending on the values of $ A _{I_1}$ and $ A _{I_2}$, we consider two cases: If they are equal, i.e. the pre-registrations differ only by a translation, we perform the shift calculation on the raw pixel data of the overlapping volume to get a shift vector $t$ for $I_2$ in pixel coordinates. The transformation in world coordinates is then given by $R \bigl(\begin{smallmatrix}  & I & & t \\  0 & \cdots & 0 & 1 \end{smallmatrix}\bigr) R^{-1} $ with $R = \bigl(\begin{smallmatrix}  & A_{I_2} & & b_{I_2} \\  0 & \cdots & 0 & 1 \end{smallmatrix}\bigr)$. If the pre-registrations differ in more than just translation, we create virtually transformed images of the smallest rectangular bounding box enclosing the overlapping volume and use them as input to the registration. As the virtual input images are already in world coordinates in this case, the resulting transformation matrix for $I_2$ is simply $\bigl(\begin{smallmatrix}  & I & & t \\  0 & \cdots & 0 & 1 \end{smallmatrix}\bigr)$

\subsubsection{Interest-point based}

For each image, we apply the current (affine) registrations to the pixel-coordinate interest points and then determine \emph{candidate point matches} via descriptor matching. We then perform model-based outlier removal via the RANSAC algorithm, yielding a set of \emph{inlier point pairs}, $C_{inliers}$, and an optimal translation $t$ for $I_2$, minimizing $\sum_{(ip_1, ip_2) \in C_{inliers} }{|| ip_1 - ip_2 - t||^2}$

\subsection{Global optimization}

\subsubsection{Estimation of globally optimal transformations}

The pairwise registration step results in \emph{links} between image (groups) $V$ (note that sice we do not use the actual image \emph{content} in the global optimization, we will refer to the images by their integer \emph{id} in this section: $V \subset \mathbb{N}$). The links can be either in the form of pairwise transformations $T^{p}$  (such that coordinates $x$ from two images $V_i$ and $V_j$ can be transformed according to $T^p_{ij} (x_{j}) = x_i$) or \emph{point correspondences} $PM$ from which such transformations can be estimated.
The pairwise registrations thus form a \emph{link graph} $(V, C)$ with edges $C = \{(i,j) \in V \times V | T^p_{ij} \in T^p \}$ between image pairs for which we could determine pairwise transformations. Simply traversing a spanning tree of the link graph and propagating the pairwise transformations can lead to the compounding of pairwise registration errors, even if the traversal is done along a \emph{minimal} spanning tree determined according to some quality metric $q_{ij}$, e.g. cross-correlation, of the pairwise registrations.

We thus make use of an algorithm for globally optimal registration by iterative minimization of square displacement of point correspondences\cite{saalfeld2010rigid} for reaching a reasonable consensus in this case. This point match-based framework allows for flexible groping and fixing of images, is applicable to, among others, time series-, chromatic channel- or view-registration and can easily be adapted to incorporate the pairwise transformations from e.g. phase correlation. The algorithm is agnostic of the transformation model (e.g. translation, affine transform,...), with the only requirement being that the model parameters can be estimated by a least-squares fit from point correspondences.

We determine the globally optimal registrations $R$ given the image (groups) $V$, pairwise links $C$, pairwise $n$-dimensional point matches $PM$ with $PM_{ij} \subset \mathbb{R}^n \times  \mathbb{R}^n$ and a set of fixed views $F \subseteq V$ by minimizing:

\begin{equation}
\label{eq:globalopt-eq1}
\argmin_{R \setminus \{R_i | V_i \in F\}} \sum_{(i,j) \in C} \bigg( \sum_{(x_{k}, y_{k}) \in PM_{ij}} || R_{i}( x_{k}) - R_{j}( y_{k}) ||^2 \bigg)
\end{equation}

Note that for all fixed views, the registration will be constrained to be the indentity transformation $I$: $\forall V_i \in F: R_{i} = I$.

\subsubsection{Global optimization given pairwise transfomations}

Using the pairwise transformations $T^{p}$ ($T^p_{ij} (x_{j}) = x_i$) between two image (groups) $V_i$ and $V_j$ given their existing registrations $R_{meta}$ and the 8-point approximate bounding box of their overlapping region $BB_{ij}$, we now estimate globally optimal \emph{accumulative transforms} $R_{acc}$. A set of image groups $F \subseteq V$ may be \emph{fixed}, in which case the registration will be constrained to be identity: $\forall i \in F: R_{acc,i} = I$. In a first pass, we consider only \emph{strong links} $C_{strong} = \{(i,j) \in V \bigtimes V | T_{p, ij} \in T_p \}$ for which we calculated valid pairwise transformations and the registrations for the images linked by them $R_{strong} = \{r_i \in R_{acc} | ((i,j) \in C_{strong} \vee (j,i) \in C_{strong}) \wedge i \notin F\}$

\subsubsection{Global optimization with iterative link dropping}

Once the global optimization terminates due to convergence or exceeding of the maximium number of iterations, we can calculate the \emph{error} of the individual images as the average displacement of all interest points in an image to their point matches:

\begin{equation}
\label{eq:gloablopt-eq2}
e_i = \frac{\sum_{\{j: (i,j) \in C\}}  \sum_{(x_{k}, y_{k}) \in PM_{ij}} || R_{i}( x_{k}) - R_{j}( y_{k}) ||  }{ \sum_{\{j: (i,j) \in C\}} |PM_{ij}| }
\end{equation}

If the link graph $(V,C^n)$ contains links with contradicting point correspondences, stopping after one round of global optimization might leave us with unsatisfying results. In the \emph{iterative} version of the global optimization, we therefore check that both the average error of all images and the ratio of maximal and average error fall below a user-defined threshold. If these condiditions are not yet met, we will proceed to iteratively remove disagreeing links from the link graph and repeat the global optimization. To do this, we first determine the link with the highest error by maximizing:

\begin{equation}
\label{eq:globalopt-eq3}
c_{worst} = \argmax_{(i,j)} \max_{(x_k, y_k) \in PM_{ij}}\bigg( (1-q_{ij})^2 \sqrt{d_{ijk}} \log_{10}\Big(\max\big(deg(i), \deg(j)\big)\Big)\bigg) 
\end{equation}

with $d_{ijk}$ denoting the distance of the $k$'th point match of the link $(i,j)$, $d_{ijk} = || R_{i}(x_k) - R_{j}(y_k) ||$, $\deg(i)$ denoting the degree (number of neighbors) of an image $V_i$ in the link graph and $q_{ij}$ being a \emph{quality metric} $\in (0,1)$ of the link, e.g. 0-truncated cross correlation. We then remove the worst link from the links ($C^{n+1} \leftarrow C^{n} \setminus c_{worst}$) and repeat the optimization step \ref{eq:globalopt-eq1} with the new link graph $(V, C^{n+1})$. The whole process is repeated until the errors fall below a user-defined threshold (in the worst case, links will be dropped until we end up with \emph{spanning trees} of the connected components in the link graph). 

\subsubsection{Two-round global optimization using metadata}

If some cases, the link graph might contain multiple connected components, e.g. in datasets from screening applications, where the actual sample only occupies isolated "islands" and most images contain only background. In this case, we can only realibly determine pairwise transformations within the connected components and align images within the components in the global optimization step. We might, however, have resonable registrations $R^{meta}$ from metadata and wish to keep as closely as possible to those if we do not have \emph{stong} links.

For this, we offer a \emph{two-round} version of the global optimization. In the first round, we determine registrations $R^{strong}$ as described above, using the graph of \emph{strong} links, i.e. links that are backed by pairwise transformations. In the second round, we determine the connected components in the $(V, C^{strong} )$ graph and a mapping $CC: \mathbb{N} \to \mathbb{N}$ from image (group) indices to  connected component indices as well as \emph{weak links} $C^{weak} = \{(i,j) \in V \times V | CC(i) \neq CC(j) \}$ between images in different components. We then determine transformations $R^{cc}$ for each connected component not containing a fixed image by minimizing:

\begin{equation}
\label{eq:globalopt-eq4}
\argmin_{R^{cc} \setminus \{ r^{cc}_{i} \in R^{cc} | CC_i \cap F \neq \emptyset \}} \sum_{(i,j) \in C^{weak}} \sum_{bb_{k} \in BB_{ij}} || R^{cc}_{CC(i)}\big( R^{strong}_{i}( bb_{k})\big) -  R^{cc}_{CC(j)}\big( R^{strong}_{j}( bb_{k}) \big) ||^2 
\end{equation}

Note that we use the corners $bb_k$ of the bounding box $BB_{ij}$ of the overlapping volume of two images $V_i$ and $V_j$ as the point correspondences. The overlap is determined accoring to the metadata transformations $R^{meta}$ and we essentially try to "un-do" the registrations of the first round as well as possible (while keeping the registrations \emph{within} the connected components). The final transformations $R$ are the concatenation of the registrations within the connected components with the relative transformations of the connected components: $R_{i} \leftarrow R^{cc}_{CC(i)} R^{strong}_{i}$.


\subsection{MultiView Registration}



\pagebreak





%================ old text, left as a comment in case we need some commands from it ====================

\begin{comment}

This document closely follows the notation introduced in the paper of L. B. Lucy\cite{lucy1974} whenever possible.
%In section \ref{sec:singleview} we re-state the formalization of the deconvolution problem as defined by Lucy and re-derive the iterative solution for the single view case as shown by Richardson\cite{richardson1972} and Lucy\cite{lucy1974}. In section \ref{sec:multiview} we derive the \emph{bayesian} extension to the multi-view case as used in\cite{KrzicPhD, PreibischPhD, Temerinac2012}. Section \ref{sec:conditional} discusses the implications of independence between views and in section \ref{sec:efficientmv} we show the derivation of the efficient \emph{bayesian} multi-view deconvolution that takes into account conditional probabilities for the first time. In section \ref{sec:optimizeconvergence} we introduce a faster alternative optimization scheme and section \ref{sec:benchmark} evaluates all methods and compares our new derivations considering conditional probabilities to previously existing methods.
Note that for simplicity all derivations in this document only cover the one-dimensional case. Nevertheless, all equations are valid for any \emph{n}-dimensional case.

%\subsection{Derivation of the iterative deconvolution scheme}
\subsection{Derivation of the Bayesian-based single-view deconvolution: REMOVE AND REPLACE WITH OUT STUFF}
\label{sec:singleview}

This section re-derives the classical Bayesian-based Richardson\cite{richardson1972}-Lucy\cite{lucy1974} deconvolution for single images, other derivations presented in this document build up on it. The goal is to estimate the frequency distribution of an underlying signal $\psi(\xi)$ from a finite number of measurements $x^{1'}, x^{2'}, ..., x^{N'}$. The resulting observed distribution $\phi(x)$ is defined as
\begin{equation}
\label{eq:eq1}
\phi(x) = \int_{\xi}{\psi(\xi)P(x|\xi)}d\xi
\end{equation}
where $P(x|\xi)$ is the probability of a measurement occuring at $x=x'$ when it is known that the event \mbox{$\xi=\xi'$} occured. In more practical image analysis terms equation \ref{eq:eq1} describes the one-dimensional convolution operation where $\phi(x)$ is the blurred image, $P(x|\xi)$ is the kernel and $\psi(\xi)$ is the undegraded (or deconvolved) image. All distributions are treated as probability distributions and fulfill the following constraints:
\begin{equation}
\label{eq:eq3}
\int_{\xi}{\psi(\xi)}d\xi = \int_{x}{\phi(x)}dx = \int_{x}{P(x|\xi)}dx = 1 ~~~and~~~ \psi(\xi) > 0, ~\phi(x) \geq 0, ~P(x|\xi) \geq 0
\end{equation}

\noindent The basis for the derivation of the Bayesian-based deconvolution is the tautology
\begin{equation}
\label{eq:eq5}
P(\xi=\xi' \wedge x=x') = P(x=x' \wedge \xi=\xi')
\end{equation}
It states that it is equally probable that the event $\xi'$ results in a measurement at $x'$ and that the measurement at $x'$ was caused by the event $\xi'$. Integrating equation \ref{eq:eq5} over the measured distribution yields the joint probability distribution
\begin{equation}
\int_x P(\xi \wedge x) dx = \int_x P(x \wedge \xi) dx
\end{equation}
which can be expressed using conditional probabilities
\begin{equation}
\int_x P(\xi) P(x|\xi) dx = \int_x P(x) P(\xi|x) dx
\end{equation}
and in correspondence to Lucy's notation looks like (equation \ref{eq:eq1})
\begin{equation}
\label{eq:eq7}
\int_x \psi(\xi) P(x|\xi) dx = \int_x \phi(x) Q(\xi|x) dx
\end{equation}
where $P(\xi) \equiv \psi(\xi), P(x) \equiv \phi(x), P(\xi|x) \equiv Q(\xi|x)$. $Q(\xi|x)$ denotes what Lucy calls the 'inverse' conditional probability to $P(x|\xi)$. It defines the probability that an event at $\xi'$ occured, given a specific measurement at $x'$. As $\psi(\xi)$ does not depend on $x$, equation \ref{eq:eq7} can be rewritten as
\begin{equation}
\label{eq:eq9}
\psi(\xi) \overbrace{\int_x P(x|\xi) dx}^{=1} = \int_x \phi(x) Q(\xi|x) dx
\end{equation}
hence (due to equation \ref{eq:eq3})
\begin{equation}
\label{eq:eq11}
\psi(\xi) = \int_x \phi(x) Q(\xi|x) dx
\end{equation}
which corresponds to the inverse of the convolution in equation \ref{eq:eq1}. Although $Q(\xi|x)$ cannot be used to directly compute $\psi(\xi)$, \emph{Bayes' Theorem} and subsequently equation \ref{eq:eq1} can be used to reformulate it as
\begin{equation}
\label{eq:eq12}
Q(\xi|x) = \frac{\psi(\xi) P(x|\xi)}{\phi(x)} =  \frac{\psi(\xi) P(x|\xi)}{\int_\xi \psi(\xi) P(x|\xi) d\xi}
\end{equation}
Replacing $Q(\xi|x)$ in equation \ref{eq:eq11} yields
\begin{equation}
\label{eq:eq13}
\psi(\xi) = \int_x \phi(x) \frac{\psi(\xi) P(x|\xi)}{\int_\xi \psi(\xi) P(x|\xi) d\xi} dx
          = \psi(\xi) \int_x  \frac{ \phi(x) }{\int_\xi \psi(\xi) P(x|\xi) d\xi} P(x|\xi) dx
\end{equation}
which exactly re-states the deconvolution scheme introduced by Lucy and Richardson. The fact that both sides of the equation contain the desired underlying (deconvolved) distribution $\psi(\xi)$ suggests an iterative scheme to converge towards the correct solution
\begin{equation}
\label{eq:eq15}
\psi^{r+1}(\xi) = \psi^r(\xi) \int_x \frac{ \phi(x) }{\int_\xi \psi^r(\xi) P(x|\xi) d\xi} P(x|\xi) dx
\end{equation}
where $\psi^0(\xi)$ is simply a constant distribution with each value being the average intensity of the measured distribution $\phi(x)$.

Equation \ref{eq:eq15} turns out to be a maximum-likelihood (ML) expection-maximization (EM) formulation\cite{Dempster77}, which works as follows. First, it computes for every pixel the convolution of the current guess of the deconvolved image $\psi^r(\xi)$ with the kernel (PSF) $P(x|\xi)$, i.e. $\phi^r(x) = \int_\xi \psi^r(\xi) P(x|\xi) d\xi$. In EM-terms $\phi^r(x)$ describes the \emph{expected value}.  The quotient between the input image $\phi(x)$ and the \emph{expected value} $\phi^r(x)$ yields the disparity for every pixel. These values are initially large but will become very small upon convergence. In an ideal scenario all values of $\phi^r(x)$ and $\phi(x)$ will be identical once the algorithm converged. This ratio is subsequently convolved with the point spread function $P(x|\xi)$ reflecting which pixels influence each other. In EM-terms this is called the \emph{maximization step}. This also preserves smoothness. These resulting values are then pixel-wise multiplied with the current guess of the deconvolved image $\psi^r(\xi)$, which we call an RL-update (Richardson-Lucy). It results in a new guess for the deconvolved image.

Starting from an initial guess of an image with constant values, this scheme will converge towards the correct solution if the guess of the point spread function is correct and if the observed distribution is not degraded by noise, transformations, etc.

pixel-by-pixel the quotient between the input image $\phi(x)$ and the convolution of the current guess of the deconvolved image $\psi^r(\xi)$ with the kernel (aka PSF) $P(x|\xi)$, which yields a per pixel difference. These values are initially large but will become very small upon convergence. This quotient image is subsequently convolved with the point spread function $P(x|\xi)$ reflecting which pixels influence each other. This also preserves smoothness. These resulting values are then pixel-wise multiplied with the current guess of the deconvolved image $\psi^r(\xi)$, yielding a new guess of the deconvolved image. Starting from an initial guess of an image with constant values, this scheme will converge towards the correct solution if the guess of the point spread function is correct and if the observed distribution is not degraded by noise, transformations, etc.


\pagebreak

\end{comment}
